\documentclass[11pt,a4wide]{article}
\usepackage{verbatim}
\usepackage{listings}
\usepackage{graphicx}
\usepackage{a4wide}
\usepackage{color}
\usepackage{amsmath}
\usepackage{amssymb}
\usepackage[dvips]{epsfig}
\usepackage[T1]{fontenc}
\usepackage{cite} % [2,3,4] --> [2--4]
\usepackage{shadow}
\usepackage{hyperref}

\setcounter{tocdepth}{2}

\lstset{language=c++}
\lstset{alsolanguage=[90]Fortran}
\lstset{basicstyle=\small}
\lstset{backgroundcolor=\color{white}}
\lstset{frame=single}
\lstset{stringstyle=\ttfamily}
\lstset{keywordstyle=\color{red}\bfseries}
\lstset{commentstyle=\itshape\color{blue}}
\lstset{showspaces=false}
\lstset{showstringspaces=false}
\lstset{showtabs=false}
\lstset{breaklines}
\begin{document}
\section*{Introduction to numerical projects}

Here follows a brief recipe and recommendation on how to write a report for each
project.
\begin{itemize}
\item Give a short description of the nature of the problem and the eventual 
numerical methods you have used.
\item Describe the algorithm you have used and/or developed. Here you may find it convenient
to use pseudocoding. In many cases you can describe the algorithm
in the program itself.

\item Include the source code of your program. Comment your program properly.
\item If possible, try to find analytic solutions, or known limits
in order to test your program when developing the code.
\item Include your results either in figure form or in a table. Remember to
       label your results. All tables and figures should have relevant captions
       and labels on the axes.
\item Try to evaluate the reliabilty and numerical stability/precision
of your results. If possible, include a qualitative and/or quantitative
discussion of the numerical stability, eventual loss of precision etc. 

\item Try to give an interpretation of you results in your answers to 
the problems.
\item Critique: if possible include your comments and reflections about the 
exercise, whether you felt you learnt something, ideas for improvements and 
other thoughts you've made when solving the exercise.
We wish to keep this course at the interactive level and your comments can help
us improve it.
\item Try to establish a practice where you log your work at the 
computerlab. You may find such a logbook very handy at later stages
in your work, especially when you don't properly remember 
what a previous test version 
of your program did. Here you could also record 
the time spent on solving the exercise, various algorithms you may have tested
or other topics which you feel worthy of mentioning.
\end{itemize}



\section*{Format for electronic delivery of report and programs}
%
The preferred format for the report is a PDF file. You can also
use DOC or postscript formats or as an ipython notebook file. 
As programming language we prefer that you choose between C/C++, Fortran2008 or Python.
The following prescription should be followed when preparing the report:
\begin{itemize}
\item Use Devilry to hand in your projects, log in  at 
\url{ http://devilry.ifi.uio.no} with your normal UiO username and password
and choose either 'fys3150' or 'fys4150'.
There you can load up the files within the deadline.
\item Upload {\bf only} the report file!  For the source code file(s) you have developed please provide us with your link to your github domain. 
The report file should include all of your discussions and a list of the codes you have developed. 
Do not include library files which are available at the course homepage, unless you have
made specific changes to them.
\item In your git repository, please include a folder which contains selected results. These can be in the form of output from your code
for a selected set of runs and input parameters. 
\item In this and all later projects, you should include tests and/or unit tests of your code(s).
\item Comments  from us on your projects, approval or not, corrections to be made 
etc can be found under
your Devilry domain and are only visible to you and the teachers of the course.

\end{itemize}

Finally, 
we encourage you to work two and two together. Optimal working groups consist of 
2-3 students. You can then hand in a common report. 


\section*{Project 3, numerical integration, deadline  October 19}

The task of this project is to integrate first in a brute force manner a six-dimensional integral which is used
to determine the ground state correlation energy between two electrons 
in a helium atom.  
The integral appears in many quantum mechanical applications.
However, if you are not too familiar with quantum mechanics, you can simply look at the mathematical details. 
We will employ both Gauss-Legendre and Gauss-Laguerre 
quadrature and Monte-Carlo integration.
Furthermore, you will need to parallelize your codes.


We assume that the wave function of each electron can be modelled like the single-particle
wave function of an electron in the hydrogen atom. The single-particle wave function  for an electron $i$ in the 
$1s$ state 
is given in terms of a dimensionless variable    (the wave function is not properly normalized)
\[
   {\bf r}_i =  x_i {\bf e}_x + y_i {\bf e}_y +z_i {\bf e}_z ,
\]
as
\[
   \psi_{1s}({\bf r}_i)  =   e^{-\alpha r_i},
\]
where $\alpha$ is a parameter and 
\[
r_i = \sqrt{x_i^2+y_i^2+z_i^2}.
\]
We will fix $\alpha=2$, which should correspond to the charge of the helium atom $Z=2$. 

The ansatz for the wave function for two electrons is then given by the product of two 
so-called 
$1s$ wave functions as 
\[
   \Psi({\bf r}_1,{\bf r}_2)  =   e^{-\alpha (r_1+r_2)}.
\]
Note that it is not possible to find a closed-form or analytical  solution to Schr\"odinger's equation for 
two interacting electrons in the helium atom. 

The integral we need to solve is the quantum mechanical expectation value of the correlation
energy between two electrons which repel each other via the classical Coulomb interaction, namely
\begin{equation}\label{eq:correlationenergy}
   \langle \frac{1}{|{\bf r}_1-{\bf r}_2|} \rangle =
   \int_{-\infty}^{\infty} d{\bf r}_1d{\bf r}_2  e^{-2\alpha (r_1+r_2)}\frac{1}{|{\bf r}_1-{\bf r}_2|}.
\end{equation}
Note that our wave function is not normalized. There is a normalization factor missing, but for this project
we don't need to worry about that.

This integral can be solved in closed form and the answer is $5\pi^2/16^2$. Can you derive this value?

\begin{enumerate}
\item[a)] Use Gauss-Legendre quadrature and compute the integral by integrating 
for each variable $x_1,y_1,z_1,x_2,y_2,z_2$ from $-\infty$ to $\infty$.
How many mesh points do you need before the results converges at the level of the third 
leading digit?  Hint:  the single-particle wave function $e^{-\alpha r_i}$  is more or less zero at
$r_i \approx ?$ (find the appropriate limit).  
You can therefore replace the integration limits $-\infty$ and $\infty$ with 
$-?$ and $?$, respectively.  You need to check that this approximation is satisfactory, that is, make a plot
of the function and check if the abovementioned limits are appropriate.
You need also to account for the potential problems which may arise when $|{\bf r}_1-{\bf r}_2|=0$.
\item[b)]   The Legendre polynomials are defined for $x\in [-1,1]$. The previous exercise gave a very unsatisfactory ad hoc procedure. We wish to improve our results. It can therefore be useful to change to another coordinate
frame
and employ the Laguerre polynomials. The Laguerre polynomials are defined for $x\in [0,\infty)$ and if we change
to spherical coordinates
\[
   d{\bf r}_1d{\bf r}_2  = r_1^2dr_1 r_2^2dr_2 dcos(\theta_1)dcos(\theta_2)d\phi_1d\phi_2,
\]
with
\[
   \frac{1}{r_{12}}= \frac{1}{\sqrt{r_1^2+r_2^2-2r_1r_2cos(\beta)}}
\]
and 
\[
cos(\beta) = cos(\theta_1)cos(\theta_2)+sin(\theta_1)sin(\theta_2)cos(\phi_1-\phi_2))
\]
we can rewrite the above integral with different integration limits. Find these limits and replace the Gauss-Legendre 
approach in a) with Laguerre polynomials.  The function gauss-laguerre.cpp can be found in the same folder as the project file.
Do your results improve? Compare with the results from a).

{\bf Important notice for c++ programmers:}, the function which computes the Gauss-Laguerre integration points and weights returns arrays which start at $1$ and end $n$ instead of the default values $0$ and $n-1$. You need to declare an array of length $n+1$. 

\item[c)] Compute the same integral but now with brute force Monte Carlo
and compare your results with those from the previous points. Discuss the differences.
With bruce force we mean that you should use the uniform distribution.
\item[d)] Improve your brute force Monte Carlo calculation by using importance sampling.
Hint: use the exponential distribution and transform to spherical coordinates.
Does the variance decrease? Does the CPU time used compared with the brute force 
Monte Carlo decrease in order to achieve the same accuracy? Comment your results
and make a list over the time each method uses. Compare the results also.
Finally, for the last exercise you should (this is optional however) 
parallelize your code using openMP and run either on your laptop or the machines at the computer laboratory. Comment these results as well. In particular, we want to see whether you achieve an optimal speed-up or not.
\end{enumerate}



\end{document}



\end{document}







