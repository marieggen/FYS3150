\documentclass[12pt]{article}
\usepackage[top=3cm, bottom=3cm, right=3cm, left=3cm]{geometry}
\usepackage{graphicx}
\usepackage{amsmath}
\usepackage{amssymb}
\usepackage{ulem}
\usepackage[english]{babel}
\usepackage[utf8]{inputenc}
\usepackage[T1]{fontenc}

\begin{document}
\title{FYS3150 - PROJECT 2 - AUTUMN 2015}
\author{Mari Dahl Eggen}
\maketitle

\newpage

\begin{flushleft}
\begin{abstract}
hei
\end{abstract}
\section*{Introduction}

\newpage
\section*{Theory}
\subsection*{Similarity transformations}
In the following we assume that the matrix $\bf A$ is a real and symmetric matrix, ${\bf A}\in \mathbb{R}^{n\times n}$. Then, there exists a set of real orthogonal matrices $\bf S_i$, such that \\
\vspace{5mm}
\begin{equation}\label{eq:similarity_transformation}
{\bf S_n^T...S_1^T A S_1...S_n} = {\bf D},
\end{equation}
\vspace{5mm}
where ${\bf S_i^T S_i = S_i S_i^T = I}$, where ${\bf I}$ is the identity matrix, and $\bf D$ is given by\\
\vspace{5mm}
$${\bf D} = 
\begin{bmatrix}
       \lambda_1 & 0 & 0 & \dots & 0          \\
       0 & \lambda_2 & 0 & \dots & 0 \\
       \vdots &  & \ddots & & \vdots \\
       0 & \dots & \dots & \dots & \lambda_n
\end{bmatrix},$$\\ 
\vspace{5mm}
where $\lambda_1, \lambda_2,...,\lambda_n$ is the eigenvalues of $\bf A$. If we look at a single similarity transformation, where $\bf B = S^T A S$ is a similarity transform of $\bf A$, we can show that $\bf B$ and $\bf A$ have the same eigenvalues, but in general not the same eigenvectors. We starts out with the eigenvalue equation of $\bf A$.\\
\vspace{5mm}
$${\bf Ax } = \lambda\bf x \quad\Rightarrow\quad {\bf S^T A I x} = \lambda {\bf S^T x}$$\\
$$\quad\Rightarrow\quad {\bf (S^T A S) (S^T x)} = \lambda {\bf (S^T x)} \quad\Rightarrow\quad {\bf B (S^T x)} = \lambda {\bf (S^T x)}$$\\
\vspace{5mm}
We then see that $\bf B$ and $\bf A$ have the same eigenvaules, but when $\bf A$ has the eigenvector $\bf x$, $\bf B$ has the eigenvector $\bf S^T x$. We then know for sure that the matrix $\bf D$ in Equation (\ref{eq:similarity_transformation}) gives the eigenvalues of $\bf A$.
\newpage
\subsection*{Jacobi's method}
This is a method one can use to obtain the eigenvalue matrix $\bf D$ in Equation (\ref{eq:similarity_transformation}).


\newpage
\section*{Method}

\section*{Results and discussion}

\section*{Conclusion}


\end{flushleft}
\end{document}